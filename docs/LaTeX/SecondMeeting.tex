\documentclass{beamer}
\beamertemplatenavigationsymbolsempty
\usepackage{tikz}


%\newtheorem*{theorem*}{Theorem}
\newtheorem{dummy}{}[section]
%\newtheorem{theorem}[dummy]{Theorem}
%\newtheorem{lemma}[dummy]{Lemma}
%\newtheorem{example}[dummy]{Example}
%\newtheorem{corollary}[dummy]{Corollary}
%\newtheorem{definition}[dummy]{Definition}
\newtheorem{remark}[dummy]{Remark}
\newtheorem{proposition}[dummy]{Proposition}
\newtheorem{observation}{Observation}
\newtheorem{question}{Question}
\newtheorem{conjecture}[dummy]{Conjecture}
\newtheorem{warning}[dummy]{Warning}
\newtheorem{exercise}[dummy]{Exercise}


\newcommand{\Z}{\mathbb{Z}}
\newcommand{\C}{\mathbb{C}}
\newcommand{\N}{\mathbb{N}}
\newcommand{\R}{\mathbb{R}}
\newcommand{\Q}{\mathbb{Q}}
\newcommand{\A}{\mathbb{A}}
\newcommand{\OO}{\mathcal{O}}
\newcommand{\LL}{\mathbf{L}}
\newcommand{\PP}{\mathcal{P}} % partitions
\newcommand{\II}{\mathcal{I}}
\newcommand{\Reg}{\mathbf{G}}

\newcommand{\coleg}{\text{co}\ell}


\newcommand{\proj}{\mathbf{P}}
\newcommand{\core}{\mathbf{core}}
\newcommand{\quot}{\mathbf{quot}}
\newcommand{\irreps}{\text{irreps}}
\newcommand{\Var}{\mathbf{Var}}
\newcommand{\Top}{\mathbf{Top}}
\newcommand{\Sur}{\mathcal{S}}
\newcommand{\cores}{\mathcal{C}}


\DeclareMathOperator{\dusty}{pdim}
\DeclareMathOperator{\Hilb}{Hilb}
\DeclareMathOperator{\Ext}{Ext}
\DeclareMathOperator{\Hom}{Hom}
\DeclareMathOperator{\DC}{DH}
\DeclareMathOperator{\Sym}{Sym}
\DeclareMathOperator{\Ker}{Ker}
\DeclareMathOperator{\cdim}{cdim}




\begin{document}


\begin{frame}{How I think of partitions: Young Diagrams}
\begin{tikzpicture}
\foreach \part [count=\i] in {4}  {
\draw (\i-1, 0) grid (\i, \part); 
\node at (\i-.5, -.5) {\part};}
\begin{scope}[xshift=3cm, yshift=1cm]
\foreach \part [count=\i] in {3,1}  {
\draw (\i-1, 0) grid (\i, \part); 
\node at (\i-.5, -.5) {\part};}
\end{scope}
\begin{scope}[xshift=6cm, yshift=2cm]
\foreach \part [count=\i] in {2,2}  {
\draw (\i-1, 0) grid (\i, \part); 
\node at (\i-.5, -.5) {\part};}
\end{scope}
\begin{scope}[xshift=2cm, yshift=-2cm]
\foreach \part [count=\i] in {2,1,1}  {
\draw (\i-1, 0) grid (\i, \part); 
\node at (\i-.5, -.5) {\part};}
\end{scope}
\begin{scope}[xshift=6cm, yshift=-2cm]
\foreach \part [count=\i] in {1,1,1,1}  {
\draw (\i-1, 0) grid (\i, \part); 
\node at (\i-.5, -.5) {\part};}
\end{scope}
\end{tikzpicture}
\end{frame}


\begin{frame}{Partitions: lists of numbers}
A \emph{partition} $\lambda$ is a nondecreasing sequence of positive integers 
 $$\lambda_1\geq \lambda_2\geq \cdots \geq \lambda_\ell>0$$ 

\begin{block}{My favoured notation}
\begin{itemize}
\item Preferred variables: $\lambda, \mu, \nu$
\item $|\lambda|=\sum \lambda_i$ is the \emph{size}
    \item The $\lambda_i$ are the \emph{parts}
    \item $\ell(\lambda)$ the number of parts is the \emph{length}
    \item $\mathcal{P}$ will denote the set of all partitions
    \item $\mathcal{P}_n$ will denote the partitions of size $n$
\end{itemize} 
\end{block}


\end{frame}

\begin{frame}{First theorems in Partitions}
\begin{theorem}[Euler Product]
$$\sum_{\lambda\in\mathcal{P}} q^{|\lambda|}t^{\ell(\lambda)}=\prod_{m\geq 1} \frac{1}{1-q^mt}$$
\end{theorem}
\begin{proof}
Expand RHS as geometric series; $(q^mt)^k$ means $k$ parts of size $m$.
\end{proof}

\begin{theorem}{Euler's Odd=Distinct} 
Let $\mathcal{O}$ be the set of partitions into \emph{odd} parts
Let $\mathcal{D}$ be the set of partitions into \emph{distinct} parts
Then $|\mathcal{O}_n|=|\mathcal{D}_n$
\end{theorem}
\begin{proof}
$$\sum_{\lambda\in\mathcal{D}}q^{|\lambda|}=\prod_{m\geq 1} (1+q^m)=\prod_{m\geq 1} \frac{1}{1-q^{2m-1}}=\sum_{\lambda\in\mathcal{O}} q^{|\lambda|}$$
\end{proof}

\end{frame}


\begin{frame}{Algebra vs. Bijections}

Euler's theorem is interesting in that it works by algebraic manipulation of power series.  It tells us the two sets $\mathcal{O}_n$ and $\mathcal{D}_n$ have the same number of elements, but it doesn't give us a \emph{bijection} between those two sets.  

\begin{block}{Exercise}

In Igor Pak's survey on partition bijections, he gives three different bijective proofs of Euler's Odd=Distinct:

\begin{itemize}
\item Sylvester's bijection
\item Glaisher's bijection
\item Iterated Dyson maps (This will be important one for us!)
\end{itemize}

Read his notes to understand how all three work; find more in depth sources for hints / to check yourself.
\end{block}

\end{frame}





\begin{frame}{Glaisher's theorem}
%Let $\mathcal{O}$ be the set of partitions with all odd parts
%Let $\mathcal{D}$ be the set of partitions with distinct parts
%\begin{theorem}$$\sum_{\lambda\in\mathcal{O}}q^{|\lambda|}=\sum_{\lambda\in\mathcal{D}}q^{|\lambda|}$$
%\end{theorem}
%\begin{proof}\begin{align*}
%\sum_{\lambda\in\mathcal{D}}q^{|\lambda|} = \prod_{m\geq 1} (1+q^m) &= \prod_{m\geq 1} %\frac{(1+q^m)(1-q^m)}{(1-q^m)} \\
%&=\prod_{m\geq 1} \frac{1-q^{2m}}{1-q^m}\\
%&=\prod_{m\geq 1} \frac{1}{1-q^{2m-1}}=\sum_{\lambda\in\mathcal{O}} q^{|\lambda|}
%\end{align*}

%\end{proof}
%\end{frame}
%\begin{frame}{Slightly more general}
\begin{itemize}
    \item $\mathcal{R}(k)$ denotes partitions with no part repeated $k$ times or more
    \item $\mathcal{D}(k)$ is the set of partitions with no part divisible by $k$
\end{itemize}
\begin{theorem}[Glaisher]
$$\sum_{\lambda\in\mathcal{R}(k)}q^{|\lambda|}=\sum_{\lambda\in\mathcal{D}(k)}q^{|\lambda|}$$
\end{theorem}
\begin{proof}$$(1+q^m+q^{2m}+\cdots q^{(k-1)m})(1-q^m)=1-q^{km}$$
\end{proof}
\begin{itemize}
\item When $k=2$ this is Euler's theorem that "Odd=Distinct"
\item But only Glaisher's bijection known to adapt?
\end{itemize}
\end{frame}

\begin{frame}{Start of "Fermionic" P.O.V.: Partitions in a box}
\begin{block}{How many partitions fit inside an $a\times b$ box?}
\begin{tikzpicture}
\draw [thin, gray] (0,0) grid (5,3);
\end{tikzpicture}

Partitions $\leftrightarrow$ lattice paths $\leftrightarrow$ binary sequences $\leftrightarrow$ binary sequences
\end{block}

\begin{block}{Now keep track of their size...}
\end{block}

\begin{block}{Now keep track of their size...}
\end{block}
\end{frame}{Setting $q=1$ gives back all the usual formulas.} 
$$[n]_q=\frac{1-q^n}{1-q}=1+q+q^2+\cdots + q^{n-1}$$ $$[n]_q!=\prod_{k=1}^n [k]_q$$
$$\binom{n}{k}_q=\frac{[n]_q!}{[k]_q![n-k]_q!}$$


\begin{theorem}[Exercise!]
Let $\mathcal{P}(a\times b)$ denote the set of partitions inside an $a\times b$ box.  
$$\sum_{\lambda\in\PP(a\times b)}q^{|\lambda|}=\binom{a+b}{a}_q$$
\end{theorem}
\begin{proof}Both sides satisfy a $q$-analog of Pascal's triangle. Induct.\end{proof} \begin{frame}{Setting $q=1$ gives back all the usual formulas.} 
$$[n]_q=\frac{1-q^n}{1-q}=1+q+q^2+\cdots + q^{n-1}$$ $$[n]_q!=\prod_{k=1}^n [k]_q$$
$$\binom{n}{k}_q=\frac{[n]_q!}{[k]_q![n-k]_q!}$$


\begin{theorem}[Exercise!]
Let $\mathcal{P}(a\times b)$ denote the set of partitions inside an $a\times b$ box.  
$$\sum_{\lambda\in\PP(a\times b)}q^{|\lambda|}=\binom{a+b}{a}_q$$
\end{theorem}
\begin{proof}Both sides satisfy a $q$-analog of Pascal's triangle. Induct.\end{proof} 
\end{frame}




\end{document}
