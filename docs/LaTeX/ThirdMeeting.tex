\documentclass{beamer}
\beamertemplatenavigationsymbolsempty
\usepackage{tikz}


%\newtheorem*{theorem*}{Theorem}
\newtheorem{dummy}{}[section]
%\newtheorem{theorem}[dummy]{Theorem}
%\newtheorem{lemma}[dummy]{Lemma}
%\newtheorem{example}[dummy]{Example}
%\newtheorem{corollary}[dummy]{Corollary}
%\newtheorem{definition}[dummy]{Definition}
\newtheorem{remark}[dummy]{Remark}
\newtheorem{proposition}[dummy]{Proposition}
\newtheorem{observation}{Observation}
\newtheorem{question}{Question}
\newtheorem{conjecture}[dummy]{Conjecture}
\newtheorem{warning}[dummy]{Warning}
\newtheorem{exercise}[dummy]{Exercise}


\newcommand{\Z}{\mathbb{Z}}
\newcommand{\C}{\mathbb{C}}
\newcommand{\N}{\mathbb{N}}
\newcommand{\R}{\mathbb{R}}
\newcommand{\Q}{\mathbb{Q}}
\newcommand{\A}{\mathbb{A}}
\newcommand{\OO}{\mathcal{O}}
\newcommand{\LL}{\mathbf{L}}
\newcommand{\PP}{\mathcal{P}} % partitions
\newcommand{\II}{\mathcal{I}}
\newcommand{\Reg}{\mathbf{G}}

\newcommand{\coleg}{\text{co}\ell}


\newcommand{\proj}{\mathbf{P}}
\newcommand{\core}{\mathbf{core}}
\newcommand{\quot}{\mathbf{quot}}
\newcommand{\irreps}{\text{irreps}}
\newcommand{\Var}{\mathbf{Var}}
\newcommand{\Top}{\mathbf{Top}}
\newcommand{\Sur}{\mathcal{S}}
\newcommand{\cores}{\mathcal{C}}


\DeclareMathOperator{\dusty}{pdim}
\DeclareMathOperator{\Hilb}{Hilb}
\DeclareMathOperator{\Ext}{Ext}
\DeclareMathOperator{\Hom}{Hom}
\DeclareMathOperator{\DC}{DH}
\DeclareMathOperator{\Sym}{Sym}
\DeclareMathOperator{\Ker}{Ker}
\DeclareMathOperator{\cdim}{cdim}




\begin{document}


\begin{frame}{Dirac's Electron Sea}

\begin{itemize}
\item Energy levels of electrons are $\Z+1/2$, e.g. 1/2, -7/2, 101/2.
\item Each energy level is either filled or empty
\item Vacuum is when every negative energy state is filled, no positive levels filled
\item A missing negative energy state corresponds to a positron with that energy
\item A \emph{state} is finite number of electrons and positrons
\end{itemize}
\begin{block}{Visualize states as \emph{Maya Diagrams}}
\begin{center}
\begin{tikzpicture}[scale=.7]
\begin{scope}[yshift=2cm]

\draw (0,-.5) to (0,.5);
\draw (-5.5,0) node{$\cdots$};
\draw (-4.5,0) circle (.3) node[below=3pt]{$\frac{9}{2}$};
\draw (-3.5,0) circle (.3) node[below=3pt]{$\frac{7}{2}$};
\draw (-2.5,0) circle (.3) node[below=3pt]{$\frac{5}{2}$};
\draw (-1.5,0) circle (.3) node[below=3pt]{$\frac{3}{2}$};
\draw (-.5,0) circle (.3) node[below=3pt]{$\frac{1}{2}$};
\filldraw (.5,0) circle (.3) node[below=3pt]{$\frac{-1}{2}$};
\filldraw (1.5,0) circle (.3) node[below=3pt]{$\frac{-3}{2}$};
\filldraw (2.5,0) circle (.3) node[below=3pt]{$\frac{-5}{2}$};
\filldraw (3.5,0) circle (.3) node[below=3pt]{$\frac{-7}{2}$};
\filldraw (4.5,0) circle (.3) node[below=3pt]{$\frac{-9}{2}$};
\draw (5.5,0) node{$\cdots$};
\end{scope}

\draw (0,-.5) to (0,.5);
\draw (-5.5,0) node{$\cdots$};
\draw (-4.5,0) circle (.3) node[below=3pt]{$\frac{9}{2}$};
\filldraw (-3.5,0) circle (.3) node[below=3pt]{$\frac{7}{2}$};
\filldraw (-2.5,0) circle (.3) node[below=3pt]{$\frac{5}{2}$};
\filldraw (-1.5,0) circle (.3) node[below=3pt]{$\frac{3}{2}$};
\draw (-.5,0) circle (.3) node[below=3pt]{$\frac{1}{2}$};
\draw (.5,0) circle (.3) node[below=3pt]{$\frac{-1}{2}$};
\filldraw (1.5,0) circle (.3) node[below=3pt]{$\frac{-3}{2}$};
\draw (2.5,0) circle (.3) node[below=3pt]{$\frac{-5}{2}$};
\filldraw (3.5,0) circle (.3) node[below=3pt]{$\frac{-7}{2}$};
\filldraw (4.5,0) circle (.3) node[below=3pt]{$\frac{-9}{2}$};
\draw (5.5,0) node{$\cdots$};
\end{tikzpicture}
\end{center}
\end{block}
\end{frame}

\begin{frame}{Charge 0, energy $n$ states $\stackrel{1:1}{\longleftrightarrow}$ partition of $n$}
\begin{center}
\begin{tikzpicture}
\begin{scope}[gray, very thin, scale=.6]
   \clip (-8.5, 8.5) rectangle (8.5, -1);
   \draw[rotate=45, scale=1.412, thin, gray, dashed] (0,0) grid (10,10);
\end{scope}
\begin{scope}[scale=.6, yshift=-.5cm]
   \draw (-8.5,0) node{$\cdots$};
     \draw (-7.5,0) circle (.3);

 \draw (-6.5,0) circle (.3);
   \draw (-5.5,0) circle (.3);

 \draw (-4.5,0) circle (.3);
   \draw (-3.5,0) circle (.3);
   \draw (-2.5,0) circle (.3);
   \draw (-1.5,0) circle (.3);
   \draw (-.5,0) circle (.3);
   \draw (.5,0) circle (.3);
   \draw (1.5,0) circle (.3);
   \draw (2.5,0) circle (.3);
   \draw (3.5,0) circle (.3);
   \draw (4.5,0) circle (.3);
   \draw (5.5,0) circle (.3);
   \draw (6.5,0) circle (.3);
    \draw (7.5,0) circle (.3);

\draw (8.5,0) node{$\cdots$};
  \end{scope}

\begin{scope}[scale=.6]
\foreach \x in {.5, 1.5,...,6.5,7.5} {
\draw[thin, dotted, gray] (\x,0) -- (\x, \x);
\draw[thin, dotted, gray] (-\x,0) -- (-\x, \x); };

\end{scope}

\end{tikzpicture}


\end{center}



\end{frame}

\begin{frame}{Cells, hooks, arms and legs in fermionic viewpoint?}
\begin{center}
\begin{tikzpicture}
\begin{scope}[gray, very thin, scale=.6]
   \clip (-8.5, 8.5) rectangle (8.5, -1);
   \draw[rotate=45, scale=1.412, thin, gray, dashed] (0,0) grid (10,10);
\end{scope}
\begin{scope}[scale=.6, yshift=-.5cm]
   \draw (-8.5,0) node{$\cdots$};
     \draw (-7.5,0) circle (.3);

 \draw (-6.5,0) circle (.3);
   \draw (-5.5,0) circle (.3);

 \draw (-4.5,0) circle (.3);
   \draw (-3.5,0) circle (.3);
   \draw (-2.5,0) circle (.3);
   \draw (-1.5,0) circle (.3);
   \draw (-.5,0) circle (.3);
   \draw (.5,0) circle (.3);
   \draw (1.5,0) circle (.3);
   \draw (2.5,0) circle (.3);
   \draw (3.5,0) circle (.3);
   \draw (4.5,0) circle (.3);
   \draw (5.5,0) circle (.3);
   \draw (6.5,0) circle (.3);
    \draw (7.5,0) circle (.3);

\draw (8.5,0) node{$\cdots$};
  \end{scope}

\begin{scope}[scale=.6]
\foreach \x in {.5, 1.5,...,6.5,7.5} {
\draw[thin, dotted, gray] (\x,0) -- (\x, \x);
\draw[thin, dotted, gray] (-\x,0) -- (-\x, \x); };

\end{scope}

\end{tikzpicture}


\end{center}



\end{frame}


\begin{frame}{Cells, hooks, arms, and legs in fermionic viewpoint!}
\begin{itemize}
\item A cell $\square\in\lambda$ is determined by its hand and foot.
\item  Given an $E$ in boundary that appears before an $S$ in boundary path, there's a cell that has that $E$ as its hand, and that $S$ as its foot. Such a pair $(E,S)$ is called an \emph{inversion}
\item In fermionic viewpoint, an inversion is a pair of energy levels $e_1>e_2$, with $e_1$ filled, and $e_2$ empty. 
\item Moving the electron with energy $e_1$ to $e_2$ corresponds to removing the rim hook of $\square$ so $h(\square)=e_1-e_2$
\item $a(\square)$ is the number of empty states between $e_1$ and $e_2$
\item $\ell(\square)$ is the number of filled states between $e_1$ and $e_2$

\end{itemize}


\end{frame}


\begin{frame}{Abacus}
To analyze hook length divisible by $k$, spread the electrons over $k$ runners:




\begin{tikzpicture}
   \begin{scope}[gray, very thin, scale=.4]
      \clip (-11.5, 11.5) rectangle (11.5, -1);
      \draw[rotate=45, scale=1.412, gray!20!white, fill] (0,0)--(0,10)--(1,10)--(1,8)--(3,8)--(3,4)--(5,4)--(5,3)--(6,3)--(6,1)--(8,1)--(8,0)--cycle;
      \begin{scope}[rotate=45, scale=1.412, black!50!green] 
         \draw (.5,6.5)node{$\spadesuit$};
         \draw (1.5,6.5)node{$\spadesuit$};
      \end{scope}

      \begin{scope}[rotate=45, scale=1.412, blue] 
         \draw (2.5,5.5)node{$\clubsuit$};
         \draw (2.5,2.5)node{$\clubsuit$};
         \draw (4.5,2.5)node{$\clubsuit$};
      \end{scope}


      \draw[rotate=45, scale=1.412] (0,0) grid (12,12);
   \end{scope}

   \begin{scope}[rotate=45, ultra thick, scale=.4*1.412]
      \begin{scope}[blue]
         \draw (1,9.5) node{$0$};
         \draw (2.5,8)node{$0$};
         \draw (3,5.5)node{$0$};
         \draw (4.5,4)node{$0$};
         \draw (6,2.5)node{$0$};
         \draw (7.5,1)node{$0$};
         \draw (9.5,0)node{$0$};
      \end{scope}

      \begin{scope}[red]
         \draw (0,10.5)node{$1$};
         \draw (1,8.5)node{$1$};
         \draw (3,7.5)node{$1$};
         \draw (3,4.5)node{$1$};
         \draw (5,3.5)node{$1$};
         \draw (6,1.5)node{$1$};
         \draw (8,.5)node{$1$};
         \draw (10.5,0)node{$1$};
      \end{scope}

      \begin{scope}[black!50!green]
         \draw (.5,10)node{$2$};
         \draw (1.5, 8)node{$2$};
         \draw (3,6.5)node{$2$};
         \draw (3.5,4)node{$2$};
         \draw (5.5,3)node{$2$};
         \draw (6.5,1)node{$2$};
         \draw (8.5,0)node{$2$};
      \end{scope}

   \end{scope}

\end{tikzpicture}


\end{frame}
\end{document}
